\documentclass[
		11pt,
		a4paper,
		openright,
		oneside,
		ngerman
	]
	{book}

\usepackage{../latexstyles/print}
\usepackage[cm]{fullpage}

\title{
	``Keynes'' ---
	A Q-Sort on the Economy and Taxation
}

\author{
	\href{http://www.maxheld.de}{Maximilian Held}
}

\date{
	\today
}

\begin{document}

\maketitle

\pagestyle{empty}

\extrarowdepth=21mm
\extrarowheight=21mm
\tabulinesep=0mm

\scriptsize

\begin{longtabu}[htpb]
	{
		m{4.5cm}
		m{4.5cm}
		m{4.5cm}
	}
%\caption[Concourse Sampling Scheme]{Concourse Sampling Scheme}
	%\label{tab:concourse-sampling-scheme}

%\tabulinesep=_2mm^2mm
%	\begin{tabular}{ lm{0.1\textwidth} cm{0.3\textwidth}cm{0.2\textwidth} rm{0.2\textwidth}}

\toprule

\multicolumn{1}{|l|}{\emph{Values / Axiology}}
&\multicolumn{1}{l|}{\emph{Beliefs / Ontology}}
&\multicolumn{1}{l|}{\emph{Preferences / Taxes}}
\\

\midrule
\endhead

\toprule

\multicolumn{1}{|l|}{\emph{Values / Axiology}}
&\multicolumn{1}{l|}{\emph{Beliefs / Ontology}}
&\multicolumn{1}{l|}{\emph{Preferences / Taxes}}
\\
\midrule
\endfoot

%Values
\textbf{C01}
	%en
		%
	%de
		Menschliche Bedürfnisse sind begrenzt.
		Jeder braucht Nahrung, Kleidung, Schutz und Krankenversorgung aber über diese und andere Lebensnotwendigkeiten hinaus brauchen wir nichts.
%Beliefs
& \textbf{C02} %T121
	%en
		%People should not be taxed on income and wealth they have earned in uncoerced exchanges with others.
		%If two people agree on a mutually beneficial deal, they are entitled to its fruits.
		%For example, if someone has a special talent and offers these services to another person, and both are --- respectively --- richer and happier as a result, then the state should not dimish these gains.
	%de
		%Man sollte Leute nicht auf das Einkommen und Vermögen besteuern, dass Sie im zwangfreien Austausch mit anderen erwirtschaftat haben.
		Wenn Leute freiwillig ein wechselseitig vorteilhaftes Geschäft abschließen, gehören ihnen die Erträge dieses Austauschs.
		Zum Beispiel: Wenn jemand ein besonderes Talent besitzt, und dieses als Dienstleistung anderen anbietet und als Resultat \emph{beide} reicher und glücklicher sind, dann sollte der Staat diese Erträge nicht mindern.

%Preferences
& \textbf{C03} %T101
	%en
		%People should be taxed at a lump-sum rate.
		%Everyone should pay the same as an absolute amount of money.
	%de
		Jeder sollte eine pauschale Pro-Kopf-Steuer bezahlen.
		Alle sollten den selben absoluten Geldbetrag bezahlen, also zum Beispiel \euro 15.000 pro Jahr.
\\

\midrule

%Values
\textbf{C04} %T114
	%en
		%Taxation should curb the power of rich and mighty corporations.
	%de
		Besteuerung sollte die Macht der großen Konzerne reduzieren.
%Beliefs
&\textbf{C05} %T122
	%en
		%
	%de
		Der ökonomische Wert einer Sache ist der Preis, den sie im Austausch mit anderen erzielt.
%Preferences
&\textbf{C06} %T102
	%en
		%Corporations should be taxed on their income.
		%If a firm makes a profit, it should pay a percentage of that in taxes.
	%de
		Kapitalgesellschaften sollten Steuern auf ihren Gewinn bezahlen.
		Zum Beispiel: Wenn eine Werft einen Profit erwirtschaftet, sollte sie einen Anteil davon als Steuern bezahlen.
\\

\midrule

%Values
\textbf{C07}
	%en
		%
	%de
		Wir haben genug Wohlstand in den Industrieländern, damit jeder zufrieden leben kann.
		Es macht keinen Sinn, noch mehr Güter und Dienstleistungen anzuhäufen.
%Beliefs
&\textbf{C08}
	%en
		%
	%de
		Der ökonomische Wert einer Sache liegt in den menschlichen Bedürfnissen, die sie befriedigt.
%Preferences
&\textbf{C09} %T103
	%en
		%People should pay taxes as a percentage of what they earn in capital and labor income.
		%For example, a person should pay a share of the rents, wages, dividends and so forth as taxes.
	%de
		Leute sollten einen Anteil ihres gesamten Einkommen aus Arbeit und Kapital als Steuern zahlen.
		Zum Beispiel sollte eine Person einen Anteil ihrer Mieteinahmen, Gehälter, Dividenden usw. als Steuern zahlen.
\\

\midrule

%Values
\textbf{C10}
	%en
		%
	%de
		Der Nutzen den wir aus mehr Gütern und Dienstleistungen ziehen können ist prinzipiell ``nach oben offen''.
		Wir können noch garnicht abschätzen, was zukünftige Technologien und Innovation für Segnungen bringen werden.
%Beliefs
&\textbf{C11}
	%en
		%
	%de
		Die relevanten sozialen Einheiten in unserer Gesellschaft sind Individueen, also einzelne Menschen wie Sie und ich.
%Preferences
&\textbf{C12} %T104
	%en
		%
	%de
		Die Marktwirtschaft sollte durch ein sozialistisches System der Produktion ersetzt werden.
		Anstelle von Wettbewerb sollten wir mehr auf demokratische Planwirtschaft setzen.
\\

\midrule

%Values
\textbf{C13} %T165
	%en
		%In taxation, only material consequences matter.
		%TODO! ADD EXAMPLE!
	%de
		Besteuerung ist eines dieser Dinge bei denen nur zählt, was am Ende rauskommt.
%Beliefs
&\textbf{C14} %T166
	%en
		%The combined value of all final goods and services is a good measure of our progress.
	%de
		Die relevanten sozialen Einheiten in unserer Gesellschaft sind Klassen, also etwa Arbeiter und Kapitalisten.
%Preferences
&\textbf{C15} %T105
	%en
		%People should pay taxes only as a percentage of their labor income, but not on the income from their savings.
		%For example, a worker should pay a share of her wages, but not on the returns from her life insurance.
	%de
		Leute sollten nur auf ihr Arbeitskommen Steuern bezahlen, aber nicht auf das Einkommen aus ihren Ersparnissen oder Vermögen.
		Zum Bespiel: Ein Arbeiterin sollte einen Anteil Steuern auf ihren Lohn bezahlen, aber nicht auf die Erträge ihrer Lebensversicherung.
\\

\midrule

%Values
\textbf{C16} %T122
	%en
		%Property and income are theft, because this surplus is forcibly extracted from someone else, somewhere else.
		%``Property'' works only if, and to the extent that government maintains it with violence.
		%For example, if someone has earned a lot of money, that person can force other people to work for her.
	%de
		Profit ist Ausbeutung.
		Zum Beispiel: Wenn ein Unternehmen einen Gewinn erwirtschaftet, dann hat das Unternehmen folglich seinen Mitarbeiterinnen und Mitarbeitern weniger als den von ihnen tatsächlich geschaffenen Wert bezahlt.
%Beliefs
&\textbf{C17} %T167
	%en
		%People react to incentives.
	%de
		Menschen reagieren auf Anreize.
		Zum Beispiel sind wir fleißig, wenn wir dafür materiell belohnt werden.
%Preferences
&\textbf{C18} %T106
	%en
		%Employee and employer should (as they do) share taxes on labor, such as social insurance contributions. For example, a worker and her employer should each pay half of the money she has to pay for health insurance.
	%de
		Arbeitnehmer und Arbeitgeber sollten je die Hälfte der Steuern auf Arbeit bezahlen.
		Zum Beispiel: Ein Bankmitarbeiter und die Bank sollten je die Hälfte der Abgaben für Krankenversicherung zahlen.
\\

\midrule

%Values
\textbf{C19} %T164
	%en
		%Taxation also serves symbolic purposes.
		%TODO: add example
	%de
		Steuern dienen auch symbolischen Zwecken.
		Zum Beispiel ist es wichtig, das Unternehmen, Arbeitnehmer und Wohlhabende gemeinsam Verantwortung übernehmen.
%Beliefs
&\textbf{C20}
	%en
		%
	%de
		Menschen sind eigentlich ziemlich soziale Wesen; sie kümmern sich um andere.
%Preferences
&\textbf{C21} %T107
	%en
		%People should pay taxes on what they consume in final goods and services at the point of sale, at a uniform rate.
		%For example, a person who goes to a restaurant should pay a tax on the receipt.
	%de
		Leute sollten Steuern zahlen als proportionalen Anteil auf die von ihnen privat konsumierten Güter und Dienstleistungen.
		Zum Beispiel sollte eine Restaurantbesucherin eine Steuer auf den Rechnungsbetrag zahlen.
\\

\midrule

%Values
\textbf{C22} %T119
	%en
		%People should pay taxes on what they take out of the economy, for example, when they buy something and use it so that no one else can use it, or so that it diminishes in value.
	%de
		Leute sollten Steuern zahlen als Anteil auf die Dinge der Ökonomie, die sie für sich selbst beanspruchen und verbrauchen.
		%Zum Beispiel sollte ein Autokäufer dafür Steuern zahlen, dass er ein Auto für sich selbst beansprucht und abnutzt, so dass das Fahrzeug an Wert verliert.
%Beliefs
&\textbf{C23} %T136
	%en
		%People often buy and consume things for others to see and admire. %T136
	%de
		Leute kaufen und konsumieren oft Dinge, damit andere sie sehen und bewundern.
%Preferences
&\textbf{C24} %T108
	%en
		%People should pay taxes on what they consume in final goods as the difference between their earnings and net savings, at a progressive rate.
		%For example, a person who earned 50k EUR, saved 20k, dissaved 10k must have consumed 40k and should be taxed on that.
		%TODO! too complicated MB
	%de
		Leute sollten Steuern zahlen als progressiven Anteil auf die von ihnen konsumierten Güter und Dienstleistungen, definiert als die Differenz zwischen ihren Einkommen und Netto-Ersparnissen.
		Zum Beispiel: Eine Person die \euro 50.000 verdient, \euro 20.000 gespart, und \euro 10.000 von Ersparnissen aufgebraucht hat, muss \euro 40.000 konsumiert haben, und sollte darauf besteuert werden.
\\

\midrule

%Values
\textbf{C25}
	%en
		%People should not have vastly different statuses.
	%de
		Es ist nicht gut, wenn Menschen in einer Gesellschaft große Statusunterschiede haben.
%Beliefs
&\textbf{C26}
	%en
		%
	%de
		Die menschliche Fähigkeit zur zentralen Planung der Ökonomie ist stark begrenzt.
		Viele dafür notwendigen Informationen ergeben sich erst in konkreten Umständen oder durch Versuch und Irrtum.
%Preferences
&\textbf{C27} %T109
	%en
		%People should pay taxes on what they own, net of all assets and liablities.
		%For example, a person who owns something worth 1500k and has debts of 200k should pay taxes on 1300k.
	%de
		Leute sollten Steuern zahlen auf ihr Nettovermögen, als Differenz zwischen ihrem Besitz und ihren Schulden.
		Zum Beispiel: Eine Person, die etwas besitzt das \euro 150.000 Wert ist, und Schulden von \euro 20.000 hat, sollte Steuern auf die verbleibenden \euro 130.000 zahlen.
\\

\midrule

%Values
\textbf{C28} %T134
	%en
		%There is something intrinsically good about people freely exchanging goods and services.
	%de
		Es ist gut und wünschenswert, wenn Leute sich selbst organisieren und frei miteinander Güter und Dienstleistungen austauschen.
%Beliefs
&\textbf{C29} %T133
	%en
		%Markets do not work very well.
	%de
		Märkte funktionieren nicht sehr gut.
%Preferences
&\textbf{C30} %T110
	%en
		%People should pay taxes on the value of natural resources they extract, but not on the value added by making them available.
		%For example, someone who digs up minerals should pay taxes on the value of those minerals as if they were still in the ground.
	%de
		Leute sollten Steuern zahlen als Anteil auf den Wert and natürlichen Rohstoffen, den Sie abbauen --- aber nicht auf den Mehrwert, den sie durch die Förderung und Bereitstellung schaffen.
		Zum Beispiel: Wenn eine Person Bodenschätze fördert, sollte sie Steuern zahlen als Anteil auf den Wert der noch ungeförderten Rohstoffe.
\\

\midrule

%Values
\textbf{C31}
	%en
		%The resources of planet earth should belong to all people.
	%de
		Die Ressourcen des Planeten Erde sollten allen Menschen gehören.
%Beliefs
&\textbf{C32}
	%en
		%Property of land is really just a function of who controls the police force in any given country.
	%de
		Niemand ``verdient'' Eigentum an Land oder andere Rohstoffen, weil diese Dinge keiner herstellt.
		Wenn eine Person Land ``besitzt'', dann heißt das eigentlich nur, dass die Polizei auf ihrer Seite ist.
%Preferences
&\textbf{C33} %T111
	%en
		%People should pay taxes on the unimproved value of land they occupy.
		%For example, someone who has build a house on a piece of land should pay taxes on the value of the land, but not the house on it, or the piping layed to it.
	%de
		Leute sollten Steuern zahlen als Anteil auf den unverbesserten Wert des Landes, dass sie belegen.
		Zum Beispiel: Wenn eine Person ein Haus auf einem Stück Land gebaut hat, dann sollte die Person Steuern zahlen auf den Wert des Landes, aber nicht den Wert des Hauses oder der Erschließung.
\\

\midrule

%Values
\textbf{C34} %T131
	%en
		%Markets are a very impoverished way to think about human motivation.
	%de
		Es beschränkt und entfremdet uns, wenn wir menschliches Handeln als Marktverhalten beschreiben.
%Beliefs
&\textbf{C35} %T130
	%en
		%What people earn on the market has quite little to do with their efforts.
	%de
		Was Leute im freien Markt verdienen hat im Allgemeinen Recht wenig mit ihren Anstrengungen zu tun.
%Preferences
&\textbf{C36} %112
	%en
		%People and corporations should pay taxes on fast, short-term, repetitive financial transactions.
		%For example, if a person or a firm bought and sold something many times a day, it should pay taxes on that revenue.
	%de
		Menschen und Unternehmen sollten Steuern zahlen auf schnelle, kurzfristige und sich oft wiederholende Finanztransaktionen.
		Zum Beispiel: Wenn eine Person oder Unternehmen mehrfach am Tag eine Aktie kauft oder verkauft, sollte sie eine Steuer auf die Summe dieser Geschäfte zahlen.
\\

\midrule

%Values
\textbf{C37} %T132
	%en
		%Markets may not be pretty, but they are our best bet to coordinate our efforts and grow the economy.
	%de
		Die Marktwirtschaft ist vielleicht nicht schön, aber für jetzt und hier ist sie wohl die beste Möglichkeit um unsere Aktivitäten zu koordinieren und unseren Wohlstand zu mehren.
%Beliefs
&\textbf{C38}
	%en
		%By and large, people earn on the market what they deserve.
	%de
		Die Marktwirtschaft entspricht unserer natürlichen Ordnung, an die wir uns halten sollten.
		Unsere Geschichte war immer eine Geschichte von Evolution und Wettbewerb.
%Preferences
&\textbf{C39} %T113
	%en
		%People and corporations should pay taxes for things that are bad for the common good.
		%For example, if a firm causes a lot of NOx pollution, it should pay a tax on that.
		%Similarly, if a person used a lot of carbon-fueled energy, she should pay a tax on that.
	%de
		Personen und Unternehmen sollten Steuern zahlen auf die Dinge, die schlecht sind für die Allgemeinheit.
		Zum Beispiel: Wenn ein Unternehmen viel Luftschadstoffe inder Produktion ausstößt, sollte es darauf Steuer zahlen.
		Auch wenn eine Person viel kohlenstoff-basierte Energie nutz, sollte sie darauf Steuern zahlen.
\\

\midrule

%Values
\textbf{C40}
	%en
		%
	%de
		Ein Wirtschaftssystem ist fair zu dem Maße, wie es den Schwächsten nutzt.
%Beliefs
&\textbf{C41} %T126
	%en
		%Much of what we take for granted in the rich world depends on taxes.
		%For example, welfare services like universal health care or unemployment insurance will not work, or work well, without taxation.
	%de
		Der Lebensstandard in den Industrieländern ist auch deshalb so hoch, weil es hohe Steuern gibt.
		Zum Beispiel basieren wolfahrtsstaatliche Errungenschaften wie Krankenversicherung und Arbeitslosenversicherung auf wirksamer Besteuerung.
%Preferences
&\textbf{C42}
	%en
		%Taxes should serve to steer markets in desirable directions.
		%For example, governments can favor industries it deems important.
	%de
		Der Staat sollte der Wirtschaft Ziele setzen, und das Marktgeschehen lenken.
		Zum Beispiel kann die Regierung einzelne Industrien besonders fördern, wenn sie für wichtig gehalten werden.
\\

\midrule

%Values
\textbf{C43}
	%en
		%
	%de
		Ein Wirtschaftssystem ist gerecht zu dem Maße, wie es Menschen Freiheit einräumt.
%Beliefs
&\textbf{C44} %T115
	%en
		%Income from capital is not really income, it is just a compensation for endured risk and delayed gratification.
		%For example, if a person does not buy a thing she wants today, but only in twenty years, the interest she gets on that saving in the meantime is just a reward for her patience and willingness to absorb the risk of, say, dying early, or the bank going under.
	%de
		Kapitalerträge sind nicht wirklich Einkommen, sie sind nur Entschädigung für das eingegangene Ausfallsrisiko und den erbrachten Belohnungsaufschub.
		Zum Beispiel: Wenn eine Person eine erwünschte Sache nicht heute kauft, sondern erst in zwanzig Jahren, dann entschädigen sie die gewonnenen Zinsen für ihre Geduld und erduldetes Risiko (etwa einer Bankenpleite oder eines frühen Todes).
%Preferences
&\textbf{C45} %T127
	%en
		%Taxation is an outmoded institution to further the common good.
		%There are better ways to provide public services, or help those in need.
		%For example, volunteer software developers can build great programs, civil society can improve schooling and charity can aid the poor.
	%de
		Besteuerung ist eine überkommene Institution um das Allgemeinwohl zu fördern.
		Es gibt heute bessere Wege um öffentliche Güter bereit zu stellen, oder den Bedürftigen zu helfen.
		Zum Beispiel schreiben freiwillige Softwareentwickler tolle Programm, die Zivilgesellschaft verbessert Schulen und Wohltätigkeit hilft den Bedürftigen.
\\

\midrule

%Values
\textbf{C46}
	%en
		%
	%de
		Was gut oder schlecht, richtig oder falsch ist lässt sich am besten in konkreten menschlichen Beziehungen beurteilen.
		Zum Beispiel sollte es uns darum gehen, für die Menschen in seinem Umfeld zu sorgen.
%Beliefs
&\textbf{C47} T116
	%en
		%Income from capital is real income, it is a rent from an owned resource that is put to use.
		%For example, if a person invests into a company, and that company turns a profit, the investment has earned the person extra resources.
	%de
		Kapitalerträge sind richtiges Einkommen; sie entstehen, wenn man mit etwas handelt, das man besitzt.
		Zum Beispiel: Wenn eine Person in ein Unternehmen investiert, und dieses Unternehmen einen Gewinn abwirft, dann hat die Investition der Person ein zusätzliches Einkommen beschert.
%Preferences
&\textbf{C48} %T147
	%en
		%Rich people should pay less in percent, but more in absolute terms than poorer people.
		%For example, a rich person may pay 800k as 10% of 8mio in taxes, but that would still be more than her fair share, compared to the 30k as 50% of 60k of a middle earner.
	%de
		Reiche Leute sollten in absoluten Summen \emph{mehr} Steuern zahlen als arme Leute, aber \emph{weniger} als arme Leute als Anteil ihres jeweiligen Einkommens.
		Zum Beispiel: Wenn eine reiche Person auf \euro 8.000.000 Jahreseinkommen 10\% Steuern zahlt, dann sind diese \euro 800.000 immer noch mehr als genug verglichen mit den \euro 30.000 Steuern, die jemand anderes als 50\% Steuern auf ein Jahreseinkommen von \euro 60.000 zahlt.
\\

\midrule

%Values
\textbf{C49} %T118
	%en
		%Corporations should not be taxed at all, because they are no moral subjects.
		%Owners, workers or customers of a firm are real people with rights and obligations, but not the firm itself.
	%de
		Kapitalunternehmen sollten keine Steuern bezahlen, weil sie keine Subjekte von Moral oder Verantwortung sind.
		Eigentümer, Mitarbeiter und Kunden eines Unternehmens sind reale Menschen mit Rechten und Pflichten, aber nicht das Unternehmen selbst.
%Beliefs
&\textbf{C50}
	%en
		%If someone has capital, that also bestows great power.
		%For example the owner of a firm has great power by making production decisions, or hiring and firing people.
	%de
		Wenn jemand viel Vermögen besitzt, dann hat er oder sie auch viel Macht.
		Zum Beispiel kann ein Unternehmer Produktionsentscheidungen treffen, oder Leute kündigen.
%Preferences
&\textbf{C51} %T128
	%en
		%Taxation is the problem, not the solution.
		%With fewer taxes, more self-reliance and competition, a lot of the problems taxation is supposed to solve would not exist in the first place.
		%For example, for-profit schools might be more responsive to the needs of all children, and everyone might find it easier to earn a living wage without the tax wedge.
	%de
		Steuern sind das Problem, nicht die Lösung.
		Mit weniger Steuern, mehr Selbstverantwortung und Wettbewerb gebe es viele der Probleme nicht, die Steuern angeblich lösen sollen.
		Zum Beispiel könnte eine profitorientierte Schule viel besser auf die Bedürfnisse aller Kinder eingehen und eine reduzierte Steuerlast würde es allen erleichtern ein menschenwürdiges Einkommen zu verdienen.
\\

\midrule

%Values
\textbf{C52} %T120
	%en
		%People should pay taxes based on their ability to pay.
		%Someone who has a large income or owns a lot of wealth can and should more of the burden.
	%de
		Leute sollten nach ihrer Leistungsfähigkeit Steuern zahlen.
		Wer viel verdient oder besitzt, kann mehr der gemeinsamen Last schultern.
%Beliefs
&\textbf{C53} %T123
	%en
		%People can be taxed on income and wealth they have earned, because society has provided the basis for these gains.
		%For example, a person may have developed a new product on public research, transported over public roads, assembled by publicy schooled workers.
	%de
		Wen man etwas verdient, verdankt man davon auch viel der Gesellschaft, die diese Gewinne ermöglicht hat.
		Zum Beispiel basiert eine erfolgreiche Erfindung vielleicht auf öffentlich finanzierter Grundlagenforschung, sie wird produziert von öffentlich ausgebildeten Facharbeitern und über öffentliche Straßen transportiert.
%Preferences
&\textbf{C54} %T145
	%en
		%Rich people should pay more as a percentage than poorer people.
		%For example, a rich person may pay 45% in taxes, and a poor person 20% of their income.
	%de
		Reiche Leute sollten \emph{mehr} als prozentualen Anteil ihres Einkommens bezahlen als arme Leute.
		Zum Beispiel könnte eine reiche Person 45\% Steuern bezahlen, eine arme Person nur 20\% ihres Einkommens.
\\

\midrule

%Values
\textbf{C55}
	%en
		%
	%de
		Es gibt manche Dinge, die sind einfach unethisch, unabhängig von den Konsequenzen.
		Zum Beispiel sollten wir keine Todesstrafe haben, selbst wenn sie Verbrechen abschrecken würde.
%Beliefs
&\textbf{C56} %T124
	%en
		%Taxation is a matter best left to experts and technocrats.
		%There's not much essentially political at stake.
		%For example, citizens may just democratically decide how much public money they want, and who should contribute how much, and leave the rest to experts.
	%de
		Steuern sind eine Sache, die man am besten den Experten überlässt.
		Es geht dabei um nichts sehr politisches.
		Wählerinnen und Wähler sollten einfach entscheiden wie viel öffentliches Geld sie ausgeben wollen, und wer dazu wieviel beitragen soll –-- und den Rest den Experten überlassen.
%Preferences
&\textbf{C57} %T146
	%en
		%Everyone should pay the same share in relative terms to their ability.
		%For example, a rich and a poor person should both a certain share (e.g. 35%) of their income.
	%de
		Jeder sollte den selben prozentualen Anteil seines Einkommens als Steuer bezahlen.
		Egal ob arm oder reich, jeder sollte einen festen Prozentsatz (etwa 35\%) seines Einkommens bezahlen.
\\

\midrule

%Values
\textbf{C58}
	%en
		%Taxes should leave markets untouched as far as possible.
		%For example, the relative sizes of industries should not change unless there is a good reason for it.
	%de
		Steuern sollten Märkte so wenig wie möglich verzerren.
		Zum Beispiel sollten sich die relativen Größen von verschiedenen Industriezweigen nicht durch eine Steuer verändern.
%Beliefs
&\textbf{C59} %T169
	%en
		%There is no objectifiable value of an activity, except for the value it has as a final good or service to consumers.
		%For example, we cannot know what the value of some basic research will be to our grandchildren, unless and until we know its commercial applications.
	%de
		Es gibt keinen objektivierbaren Wert einer Aktivität, außer dessen Wert als Gut oder Dienstleistung für private Konsumenten.
		Zum Beispiel können wir den Wert (oder Unsinn) von Grundlagenforschung für unsere Urenkel erst wissen, wenn wir dessen (eventuelle) kommerzielle Anwendungen kennen.
%Preferences
&\textbf{C60} %T170
	%en
		%By and large, consumers know best, what is good for them.
	%de
		Im Groben und Ganzen wissen Konsumenten selbst am besten, was für sie gut ist.
\\

\midrule

%Values
\textbf{C61} %T148
	%en
		%Rich people already pay more than their fair share.
	%de
		Reiche Leute zahlen schon mehr als ihren gerechten Anteil.
%Beliefs
&\textbf{C62} %T137
	%en
		%We are saving enough for our own, and our childrens future.
		%For example, the buildings we build, and the technology we develop will serve us for generations to come.
		%TODO! Mandy: unfitting example; agree with the statement, but not with the example and vice versa
	%de
		Wir sparen genug für unsere eigene Zukunft, und die Zukunft unserer Kinder.
		Bisher sind wir noch immer weiter gewachsen.
%Preferences
&\textbf{C63} %T158
	%en
		%Taxation should also consider unpaid work that people do.
		%For example, caregivers and homemakers should enjoy special privileges.
	%de
		Steuern sollten auch die unbezahlte Arbeit berücksichtigen, die Leute leisten.
		Zum Beispiel sollten pflegende Angehörige, Hausfrauen und Hausmänner besser gestellt werden.
\\

\midrule

%Values
\textbf{C64}
	%en
		%
	%de
		Ein Steuersystem sollte vor allem einfach und verständlich sein.
%Beliefs
&\textbf{C65} %T138
	%en
		%We are not saving enough for our own, and our childrens' future.
		%For example, we are causing costly global warming and amassing public debt.
	%de
		Wir sparen nicht genug für unsere eigene Zukunft, und die Zukunft unserer Kinder.
		Zum Beispiel müssen wir zukünftig große Kosten des Klimawandels tragen, und riesige Schuldenberge des Staats abtragen.
%Preferences
&\textbf{C66} %T157
	%en
		%People should be taxed on the imputed income from owner-occupied housing.
	%de
		Leute sollten auf die entgangenen Mieteinnahmen von ihrem selbstgenutztem Wohneigentum besteuert werden.
		Zum Beispiel sollte eine Person, die in ihrem eigenen Haus wohnt, Steuern zahlen auf die angenommenen Mieteinnahmen, die sie sonst mit dem Haus erwirtschaften könnte.
\\

\midrule

%Values
\textbf{C67}
	%en
		%
	%de
		Steuern sollten ungleich gestellte auch ungleich behandeln.
%Beliefs
&\textbf{C68} %T150
	%en
		%Middle earners currently bear too much of the tax burden.
	%de
		%
%Preferences
&\textbf{C69} %T139
	%en
		%Overall, people should be taxed less and left with more resources to spend or invest themselves.
	%de
		%
\\

\midrule

%Values
\textbf{C70} %T152
	%en
		%Growth is more important than equality.
	%de
		Wachstum ist wichtiger als Gleichheit.
%Beliefs
&\textbf{C71} %T155
	%en
		%Growth is an empty concept.
		%TODO: What does this mean? it is not well-defined? wrong?
	%de
		Ökonomisches Wachstum ist ein leeres Konzept.
%Preferences
&\textbf{C72} %T140
	%en
		%Overall, people should be taxed more and sacrifice more of their own spending and investment.
	%de
		Im Allgemeinen sollten die Leute höhere Steuern zahlen, und mehr ihres eigenen Konsums und ihrer Investitionen für die Allgemeinheit opfern.
\\

\midrule

%Values
\textbf{C73}
	%en
		%
	%de
		Steuern sollten gleich gestellte auch gleich behandeln.
%Beliefs
&\textbf{C74} %T141
	%en
		%People might change their economic behavior, if their income is taxed.
		%They might not try as hard to earn money as they would without an income tax.
		%For example, a worker might cut her hours, or an investor might decide not to fund a risky project.
	%de
		Leute werden ihr Verhalten ändern, wenn ihr Einkommen besteuert wird.
		Vielleicht werden sie sich weniger anstrengen um Geld zu verdienen, als sie es mit einer geringeren Steuerlast getan hätten.
		Zum Beispiel könnte eine Arbeiterin ihre Arbeitsstunden reduzieren, oder ein Investor könnte ein riskantes Projekt meiden.
%Preferences
&\textbf{C75} %T156
	%en
		%People should be taxed on the imputed income from the leisure they enjoy.
	%de
		Leute sollten besteuert werden auf das ihnen in ihrer Freizeit entgangene Einkommen.
		Zum Beispiel sollte jemand, der jeden Abend zwei Stunden Freizeit genießt, darauf so besteuert werden, als ob er in der Zeit für einen Lohn gearbeitet hätte.
\\

\midrule

%Values
\textbf{C76} %T159
	%en
		%The tax system should be as simple as possible, with no extra tax credits, subsidies or privileges.
	%de
		%
%Beliefs
&\textbf{C77} %T142
	%en
		%People will not change their behavior much, if their income is taxed.
		%They do the things they do for many reasons, not just for money.
		%For example, a worker might also be concerned about respect from her colleagues, or an investor might enjoy developing a risky project.
	%de
		Leute werden ihr Verhalten nicht sehr ändern, wenn ihr Einkommen mehr oder weniger besteuert wird.
		Leute arbeiten und investieren aus vielen Gründen, nicht nur wegen des Geldes.
		Zum Beispiel mag es einem Arbeiter auch um den Respekt seiner Kollegen gehen, oder einer Investorin mag an den Erfolg eines riskanten Projektes glauben.
%Preferences
&%empty, last cell
\\

\midrule

\center{\textbf{Stimme voll zu}}
&\center{\textbf{Stimme überhaupt nicht zu}}
&%empty, last cell

\end{longtabu}

\end{document}