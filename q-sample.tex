\documentclass[
		11pt,
		a4paper,
		openright,
		oneside,
		ngerman
	]
	{book}

\usepackage{../latexstyles/print}
\usepackage[cm]{fullpage}

\title{
	``Keynes'' ---
	A Q-Sort on the Economy and Taxation
}

\author{
	\href{http://www.maxheld.de}{Maximilian Held}
}

\date{
	\today
}

\begin{document}

\maketitle

\pagestyle{empty}

\extrarowdepth=21mm
\extrarowheight=21mm
\tabulinesep=0mm

\begin{longtabu}[htpb]
	{
		m{4.5cm}
		m{4.5cm}
		m{4.5cm}
	}
%\caption[Concourse Sampling Scheme]{Concourse Sampling Scheme}
	%\label{tab:concourse-sampling-scheme}

%\small

%\tabulinesep=_2mm^2mm
%	\begin{tabular}{ lm{0.1\textwidth} cm{0.3\textwidth}cm{0.2\textwidth} rm{0.2\textwidth}}

\toprule

\multicolumn{1}{|l|}{\emph{Values / Axiology}}
&\multicolumn{1}{l|}{\emph{Beliefs / Ontology}}
&\multicolumn{1}{l|}{\emph{Preferences / Taxes}}
\\

\midrule
\endhead

\toprule

\multicolumn{1}{|l|}{\emph{Values / Axiology}}
&\multicolumn{1}{l|}{\emph{Beliefs / Ontology}}
&\multicolumn{1}{l|}{\emph{Preferences / Taxes}}
\\
\midrule
\endfoot

%Values
\textbf{C01}
	%en
		%
	%de
		%
%Beliefs
& \textbf{C02} %T121
	%en
		%People should not be taxed on income and wealth they have earned in uncoerced exchanges with others. If two people agree on a mutually beneficial deal, they are entitled to its fruits.
		%For example, if someone has a special talent and offers these services to another person, and both are --- respectively --- richer and happier as a result, then the state should not dimish these gains.
	%de
		%
%Preferences
& \textbf{C03} %T101
	%en
		%People should be taxed at a lump-sum rate.
		%Everyone should pay the same as an absolute amount of money.
	%de
		%
\\

\midrule

%Values
\textbf{C04} %T114
	%en
		%Taxation should curb the power of rich and mighty corporations.
	%de
		%
%Beliefs
&\textbf{C05} %T122
	%en
		%
	%de
		%
%Preferences
&\textbf{C06} %T102
	%en
		%Corporations should be taxed on their income.
		%If a firm makes a profit, it should pay a percentage of that in taxes.
	%de
		%
\\

\midrule

%Values
\textbf{C07}
	%en
		%
	%de
		%
%Beliefs
&\textbf{C08}
	%en
		%
	%de
		%
%Preferences
&\textbf{C09} %T103
	%en
		People should pay taxes as a percentage of what they earn in capital and labor income.
		For example, a person should pay a share of the rents, wages, dividends and so forth as taxes.
	%de
		%
\\

\midrule

%Values
\textbf{C10}
	%en
		%
	%de
		%
%Beliefs
&\textbf{C11}
	%en
		%
	%de
		%
%Preferences
&\textbf{C12} %T104
	%en
		Local businesses should pay taxes to the local community in relation to the revenue generated in that locality.
	%de
		%
\\

\midrule

%Values
\textbf{C13} %T165
	%en
		%In taxation, only material consequences matter.
	%de
		%
%Beliefs
&\textbf{C14} %T166
	%en
		%The combined value of all final goods and services is a good measure of our progress.
	%de
		%
%Preferences
&\textbf{C15} %T105
	%en
		People should pay taxes only as a percentage of their labor income, but not on the income from their savings.
		For example, a worker should pay a share of her wages, but not on the returns from her life insurance.
	%de
		%
\\

\midrule

%Values
\textbf{C16} %T122
	%en
		%Property and income are theft, because this surplus is forcibly extracted from someone else, somewhere else.
		%``Property'' works only if, and to the extent that government maintains it with violence.
		%For example, if someone has earned a lot of money, that person can force other people to work for her.
	%de
		%
%Beliefs
&\textbf{C17} %T167
	%en
		%People react to incentives.
	%de
		%
%Preferences
&\textbf{C18} %T106
	%en
		%Employee and employer should share taxes on labor, such as social insurance contributions. For example, a worker and her employer should each pay half of the money she has to pay for health insurance.
	%de
		%
\\

\midrule

%Values
\textbf{C19} %T164
	%en
		%Taxation also serves symbolic purposes.
	%de
		%
%Beliefs
&\textbf{C20}
	%en
		%
	%de
		%
%Preferences
&\textbf{C21} %T107
	%en
		%People should pay taxes on what they consume in final goods and services at the point of sale, at a uniform rate.
		%For example, a person who goes to a restaurant should pay a tax on the receipt.
	%de
		%
\\

\midrule

%Values
\textbf{C22} %T119
	%en
		%People should pay taxes on what they take out of the economy, for example, when they buy something and use it so that no one else can use it, or so that it diminishes in value.
	%de
		%
%Beliefs
&\textbf{C23} %T136
	%en
		%People often buy and consume things for others to see and admire. %T136
	%de
		%
%Preferences
&\textbf{C24} %T108
	%en
		%People should pay taxes on what they consume in final goods as the difference between their earnings and net savings, at a progressive rate.
		%For example, a person who earned 50k EUR, saved 20k, dissaved 10k must have consumed 40k and should be taxed on that.
	%de
		%
\\

\midrule

%Values
\textbf{C25}
	%en
		%People should have vastly different statuses.
	%de
		%
%Beliefs
&\textbf{C26} %T135
	%en
		%People mostly buy and consume things because they find them intrinsically satisfying.
	%de
		%
%Preferences
&\textbf{C27} %T109
	%en
		%People should pay taxes on what they own, net of all assets and liablities.
		%For example, a person who owns something worth 1500k and has debts of 200k should pay taxes on 1300k.
	%de
		%
\\

\midrule

%Values
\textbf{C28} %T134
	%en
		%There is something intrinsically good about people freely exchanging goods and services.
	%de
		%
%Beliefs
&\textbf{C29} %T133
	%en
		%Markets do not work very well.
	%de
		%
%Preferences
&\textbf{C30} %T110
	%en
		%People should pay taxes on the value of natural resources they extract, but not on the value added by making them available.
		%For example, someone who digs up minerals should pay taxes on the value of those minerals as if they were still in the ground.
	%de
		%
\\

\midrule

%Values
\textbf{C31}
	%en
		%The resources of planet earth should belong to all people.
	%de
		%
%Beliefs
&\textbf{C32}
	%en
		%Property of land is really just a function of who controls the police force in any given country.
	%de
		%
%Preferences
&\textbf{C33} %T111
	%en
		%People should pay taxes on the unimproved value of land they occupy.
		%For example, someone who has build a house on a piece of land should pay taxes on the value of the land, but not the house on it, or the piping layed to it.
	%de
		%
\\

\midrule

%Values
\textbf{C34} %T131
	%en
		%Markets are a very impoverished way to think about human motivation.
	%de
		%
%Beliefs
&\textbf{C35} %T130
	%en
		%What people earn on the market has quite little to do with their efforts.
	%de
		%
%Preferences
&\textbf{C36} %112
	%en
		%People and corporations should pay taxes on fast, short-term, repetitive financial transactions.
		%For example, if a person or a firm bought and sold something many times a day, it should pay taxes on that revenue.
	%de
		%
\\

\midrule

%Values
\textbf{C37} %T132
	%en
		%Markets may not be pretty, but they are our best bet to coordinate our efforts and grow the economy.
	%de
		%
%Beliefs
&\textbf{C38} %T129
	%en
		%By and large, people earn on the market what they deserve.
	%de
		%
%Preferences
&\textbf{C39} %T113
	%en
		%People and corporations should pay taxes for things that are bad for the common good.
		%For example, if a firm causes a lot of NOx pollution, it should pay a tax on that.
		%Similarly, if a person used a lot of carbon-fueled energy, she should pay a tax on that.
	%de
		%
\\

\midrule

%Values
\textbf{C40}
	%en
		%
	%de
		%
%Beliefs
&\textbf{C41} %T126
	%en
		%Much of what we take for granted in the rich world depends on taxes.
		%For example, welfare services like universal health care or unemployment insurance will not work, or work well, without taxation.
	%de
		%
%Preferences
&\textbf{C42}
	%en
		%Taxes should serve to steer markets in desirable directions.
		%For example, governments can favor industries it deems important.
	%de
		%
\\

\midrule

%Values
\textbf{C43}
	%en
		%
	%de
		%
%Beliefs
&\textbf{C44} %T115
	%en
		%Income from capital is not really income, it is just a compensation for endured risk and delayed gratification.
		%For example, if a person does not buy a thing she wants today, but only in twenty years, the interest she gets on that saving in the meantime is just a reward for her patience and willingness to absorb the risk of, say, dying early, or the bank going under.
	%de
		%
%Preferences
&\textbf{C45} %T127
	%en
		%Taxation is an outmoded institution to further the common good.
		%There are better ways to provide public services, or help those in need.
		%For example, volunteer software developers can build great programs, civil society can improve schooling and charity can aid the poor.
	%de
		%
\\

\midrule

%Values
\textbf{C46}
	%en
		%
	%de
		%
%Beliefs
&\textbf{C47} T116
	%en
		%Income from capital is real income, it is a rent from an owned resource that is put to use.
		%For example, if a person invests into a company, and that company turns a profit, the investment has earned the person extra resources.
	%de
		%
%Preferences
&\textbf{C48} %T147
	%en
		%Rich people should pay less in percent, but more in absolute terms than poorer people.
		%For example, a rich person may pay 800k as 10% of 8mio in taxes, but that would still be more than her fair share, compared to the 30k as 50% of 60k of a middle earner.
	%de
		%
\\

\midrule

%Values
\textbf{C49} %T118
	%en
		%Corporations should not be taxed at all, because they are no moral subjects.
		%Owners, workers or customers of a firm are real people with rights and obligations, but not the firm itself.
	%de
		%
%Beliefs
&\textbf{C50}
	%en
		%If someone has capital, that also bestows great power.
		%For example the owner of a firm has great power by making production decisions, or hiring and firing people.
	%de
		%
%Preferences
&\textbf{C51} %T128
	%en
		%Taxation is the problem, not the solution.
		%With fewer taxes, more self-reliance and competition, a lot of the problems taxation is supposed to solve would not exist in the first place.
		%For example, for-profit schools might be more responsive to the needs of all children, and everyone might find it easier to earn a living wage without the tax wedge.
	%de
		%
\\

\midrule

%Values
\textbf{C52} %T120
	%en
		%People should pay taxes based on their ability to pay.
		%Someone who has a large income or owns a lot of wealth can and should more of the burden.
	%de
		%
%Beliefs
&\textbf{C53} %T123
	%en
		%People can be taxed on income and wealth they have earned, because society has provided the basis for these gains.
		%For example, a person may have developed a new product on public research, transported over public roads, assembled by publicy schooled workers.
	%de
		%
%Preferences
&\textbf{C54} %T145
	%en
		%Rich people should pay more as a percentage than poorer people.
		%For example, a rich person may pay 45% in taxes, and a poor person 20% of their income.
	%de
		%
\\

\midrule

%Values
\textbf{C55}
	%en
		%
	%de
		%
%Beliefs
&\textbf{C56} %T124
	%en
		%Taxation is a matter best left to experts and technocrats.
		%There's not much essentially political at stake.
		%For example, citizens may just democratically decide how much public money they want, and who should contribute how much, and leave the rest to experts.
	%de
		%
%Preferences
&\textbf{C57} %T146
	%en
		%Everyone should pay the same share in relative terms to their ability.
		%For example, a rich and a poor person should both a certain share (e.g. 35%) of their income.
	%de
		%
\\

\midrule

%Values
\textbf{C58}
	%en
		%Taxes should leave markets untouched as far as possible.
		%For example, the relative sizes of industries should not change unless there is a good reason for it.
	%de
		%
%Beliefs
&\textbf{C59} %T169
	%en
		%There is no objectifiable value of an activity, except for the value it has as a final good or service to consumers.
		%For example, we cannot know what the value of some basic research will be to our grandchildren, unless and until we know its commercial applications.
	%de
		%
%Preferences
&\textbf{C60} %T170
	%en
		%By and large, consumers know best, what is good for them.
	%de
		%
\\

\midrule

%Values
\textbf{C61} %T148
	%en
		%Rich people already pay more than their fair share.
	%de
		%
%Beliefs
&\textbf{C62} %T137
	%en
		%We are saving enough for our own, and our childrens future.
		%For example, the buildings we build, and the technology we develop will serve us for generations to come.
	%de
		%
%Preferences
&\textbf{C63} %T158
	%en
		%Taxation should also consider unpaid work that people do.
		%For example, caregivers and homemakers should enjoy special privileges.
	%de
		%
\\

\midrule

%Values
\textbf{C64}
	%en
		%
	%de
		%
%Beliefs
&\textbf{C65} %T138
	%en
		%We are not saving enough for our own, and our childrens' future.
		%For example, we are causing costly global warming and amassing public debt.
	%de
		%
%Preferences
&\textbf{C66} %T157
	%en
		%People should be taxed on the imputed income from owner-occupied housing.
	%de
		%
\\

\midrule

%Values
\textbf{C67}
	%en
		%
	%de
		%
%Beliefs
&\textbf{C68} %T150
	%en
		%Middle earners currently bear too much of the tax burden.
	%de
		%
%Preferences
&\textbf{C69} %T139
	%en
		%Overall, people should be taxed less and left with more resources to spend or invest themselves.
	%de
		%
\\

\midrule

%Values
\textbf{C70} %T152
	%en
		%Growth is more important than equality.
	%de
		%
%Beliefs
&\textbf{C71}
	%en
		%Growth is an empty concept.
	%de
		%
%Preferences
&\textbf{C72} %T140
	%en
		%Overall, people should be taxed more and sacrifice more of their own spending and investment.
	%de
		%
\\

\midrule

%Values
\textbf{C73}
	%en
		%
	%de
		%
%Beliefs
&\textbf{C74} %T141
	%en
		%People might change their economic behavior, if their income is taxed.
		%They might not try as hard to earn money as they would without an income tax.
		%For example, a worker might cut her hours, or an investor might decide not to fund a risky project.
	%de
		%
%Preferences
&\textbf{C75} %T156
	%en
		%People should be taxed on the imputed income from the leisure they enjoy.
	%de
		%
\\

\midrule

%Values
\textbf{C76} %T159
	%en
		%The tax system should be as simple as possible, with no extra tax credits, subsidies or privileges.
	%de
		%
%Beliefs
&\textbf{C77} %T142
	%en
		%People will not change their behavior much, if their income is taxed.
		%They do the things they do for many reasons, not just for money.
		%For example, a worker might also be concerned about respect from her colleagues, or an investor might enjoy developing a risky project.
	%de
		%
%Preferences
&%empty, last cell
\\

\end{longtabu}

\end{document}